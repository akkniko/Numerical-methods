\documentclass[12pt]{article}
\usepackage{graphicx}
\usepackage[russian]{babel}
\usepackage[a4paper, total={7in, 10in}]{geometry}
\usepackage{verbatim}
\usepackage{listings}
\graphicspath{ {./images/} }
\lstset{
  basicstyle=\ttfamily,
  mathescape
}
\begin{document}
\thispagestyle{empty}
\begin{center}
\large
Санкт-Петербургский Политехнический Университет 
    
Высшая школа прикладной математики и вычислительной физики, ФизМех

01.03.02 Прикладная математика и информатика

\end{center}

\vspace{4cm}

\begin{center}
\Large

\textbf{Отчет по лабораторной работе № 2}

\textbf{"Интерполяция табличных функций сплайнами"}

\textbf{дисциплина "Численные методы"}

\end{center}

\vspace{11cm}
\noindent%
\begin{minipage}[t]{.75\textwidth}
\Large
Выполнил студент гр. 5030102/30003

Преподаватель
\end{minipage}%
\begin{minipage}[t]{.5\textwidth}
\Large
 Никоноров А. К.

 Музалевский А.В.
\end{minipage}%
\vspace{3cm}

\begin{center}
\large
\today
\end{center}
\newpage
        
\newpage


\subsection*{1. Формулировка задачи}
Интерполирование табличной функции представляет собой ключевую задачу численных методов, позволяющую вычислять значения функции в промежуточных точках и заменять сложные зависимости полиномиальными приближениями для упрощения интегрирования, дифференцирования и других математических операций, при этом обеспечивая достаточную близость интерполяционного полинома к исходной функции.

\section*{2. Описание метода}
Даны узлы \( \{x_i\}_{i=0}^n \) (сетка \( x^h \)) и соответствующие значения функции \( \{y_i\}_{i=0}^n \) (сеточная функция \( y^h \)), образующие табличную функцию \( (x^h, y^h) \). Требуется построить интерполяционную функцию \( \phi(x) \) в виде кубического сплайна на отрезке \([a, b]\), используя равномерную и чебышевскую сетки, такую, что:
\[
\phi(x) \approx (x^h, y^h).
\]

\subsection*{2.1. Кубические сплайны}
Кубический сплайн \( S(x) \) для сетки \( \{x_i\}_{i=0}^n \) — это кусочно-полиномиальная функция степени 3, удовлетворяющая следующим условиям:
\begin{itemize}
    \item \( S(x_i) = y_i \) для всех \( i = 0, \dots, n \) (условие интерполяции);
    \item \( S(x) \), \( S'(x) \) и \( S''(x) \) непрерывны на \([a, b]\);
    \item На каждом интервале \([x_i, x_{i+1}]\), \( S(x) \) является полиномом третьей степени.
\end{itemize}
Для построения \( S(x) \) используются дополнительные условия, например, \( S''(a) = S''(b) = 0 \).

\subsection*{2.2. Условия применимости}
Метод применим при наличии узлов интерполяции и значений функции в этих узлах. Для повышения точности может потребоваться учет производных.


\section*{3. Тестовый пример с расчетами для задачи малой размерности}

Рассмотрим пример построения интерполяционного кубического сплайна для малой сетки. Заданы узлы \( x_0 = 0 \), \( x_1 = 1 \), значения функции \( y_0 = 1 \), \( y_1 = 0 \) и производные \( y'_0 = 2 \), \( y'_1 = -1 \).

Для двух узлов кубический сплайн \( S(x) \) на интервале \([0, 1]\) представляется одним полиномом третьей степени:
\[
S(x) = a x^3 + b x^2 + c x + d,
\]
где первая производная:
\[
S'(x) = 3a x^2 + 2b x + c.
\]
Сплайн должен удовлетворять следующим условиям:
\begin{itemize}
    \item \( S(0) = y_0 = 1 \),
    \item \( S(1) = y_1 = 0 \),
    \item \( S'(0) = y'_0 = 2 \),
    \item \( S'(1) = y'_1 = -1 \).
\end{itemize}

Составим систему уравнений:

1. Для \( S(0) = 1 \):
\[
S(0) = d = 1 \quad \Rightarrow \quad d = 1.
\]

2. Для \( S(1) = 0 \):
\[
S(1) = a (1)^3 + b (1)^2 + c (1) + d = a + b + c + d = 0.
\]
Подставим \( d = 1 \):
\[
a + b + c + 1 = 0 \quad \Rightarrow \quad a + b + c = -1.
\]

3. Для \( S'(0) = 2 \):
\[
S'(0) = c = 2 \quad \Rightarrow \quad c = 2.
\]

4. Для \( S'(1) = -1 \):
\[
S'(1) = 3a (1)^2 + 2b (1) + c = 3a + 2b + c = -1.
\]
Подставим \( c = 2 \):
\[
3a + 2b + 2 = -1 \quad \Rightarrow \quad 3a + 2b = -3.
\]

Теперь у нас есть:
\[
a + b + c = -1, \quad 3a + 2b = -3, \quad c = 2, \quad d = 1.
\]
Подставим \( c = 2 \) в первое уравнение:
\[
a + b + 2 = -1 \quad \Rightarrow \quad a + b = -3.
\]

Решаем систему двух уравнений:
\[
a + b = -3, \quad (1)
\]
\[
3a + 2b = -3. \quad (2)
\]
Умножим уравнение (1) на 2:
\[
2a + 2b = -6. \quad (3)
\]

Коэффициенты:
\[
a = 3, \quad b = -6, \quad c = 2, \quad d = 1.
\]

Тогда кубический сплайн:
\[
S(x) = 3x^3 - 6x^2 + 2x + 1.
\]



\subsection*{4 Результаты}
Для функций \( y = x^2 - \sin(10x) \) и \(y = x+100|x|\) была построена зависимость максимальной ошибки интерполяции от количества узлов.

\begin{figure}[h!] % h! - попытка разместить картинку именно здесь
    \centering % центрирование картинки
    \includegraphics[width=0.7\linewidth]{1.jpg} % путь к картинке и её ширина
    \caption{Зависимость ошибки интерполяции от размера сетки для \( y = x^2 - \sin(10x) \)}
    \label{fig:error} % метка для ссылок
\end{figure}

\begin{figure}[h!] % h! - попытка разместить картинку именно здесь
    \centering % центрирование картинки
    \includegraphics[width=0.7\linewidth]{2.jpg} % путь к картинке и её ширина
    \caption{Зависимость ошибки интерполяции от размера сетки для \( y=x+100|x| \)} % подпись к картинке
    \label{fig:error} % метка для ссылок
\end{figure}


\newpage
\\Из графиков видно, что равномерная сетка хуже справляется с интерполированием
исходных функций. При увеличении числа узлов равномерной сетки максимальная ошибка функции показывает большую погрешность по сравнению с Чебышевской сеткой. 
\subsection*{5 Выводы}
\par 1. Гладкость функции влияет на точность интерполяции.
\\2. Увеличение количества узлов сетки способствует уменьшению максимальной ошибки. 
\\3. На основании вышеприведенного можно сделать вывод, что для увеличения точности при построении полинома стоит пользоваться Чебышёвской сеткой, а метод сплайнов эффективен для интерполяции функций


\end{document}